\documentclass[12pt, a4paper]{article}
\usepackage[utf8]{inputenc}
\usepackage[T1]{fontenc,url}
\usepackage{babel,textcomp}
\usepackage{amsmath,amsfonts,amssymb,amscd,amsthm,xspace}
\usepackage{polynom}
\usepackage[parfill]{parskip}
% ------------------------------------------------------------
% Side formatering
\usepackage{fancyhdr}
\lhead{Financial Mathematics}
\rhead{Uy Quoc Nguyen}
\usepackage{lastpage}
\cfoot{Page \thepage}
% ------------------------------------------------------------
% ------------------------------------------------------------
% \theoremstyle{plain}
\newtheorem{example}{Example}[section]
\newtheorem{theorem}{Theorem}[section]
\newtheorem*{theorem*}{Theorem}
\newtheorem{corollary}[theorem]{Corollary}
\newtheorem{lemma}[theorem]{Lemma}
\newtheorem{proposition}[theorem]{Proposition}
\newtheorem{axiom}[theorem]{Axiom}
\theoremstyle{definition}
\newtheorem{definition}[theorem]{Definition}
\theoremstyle{remark}
\newtheorem{remark}[theorem]{Remark}
% ------------------------------------------------------------

% ------------------------------------------------------------
% Egne kommandoer for matematiske uttrykk:
\newcommand{\variance}[1][X]{\mathrm{Var}\left(#1\right)}
\newcommand{\expected}[1][X]{\mathrm{E}\left(#1\right)}
% ------------------------------------------------------------

\title
{
	Project TuttiFrutti\\
	Financial Mathematics
}
\author{Uy Quoc Nguyen}
\date{\today}

\begin{document}
	\maketitle
	\pagenumbering{gobble}
	\clearpage
	\setcounter{tocdepth}{0}
	\tableofcontents
	\clearpage
	\pagestyle{fancy}
	\pagenumbering{arabic}
	\chead{Linear Algebra}
\part{Linear Algebra}
\section{Linear transformation}
\begin{definition}[Linear transformation]
	Let $V$ and $W$ be a vector spaces over some field $\mathbb F$. Then a function $T\colon V\to W$ is a \emph{linear transformation} if 
	$$
	T(a\vec{v} + b\vec{u}) = aT(\vec{v}) + bT(\vec{u})
	$$
\end{definition}
Hence, for an $m\times n$ matrix $A$. We have that $A$ defines a linear transformation from $\mathbb R^n$ to $\mathbb R^m$, written as $A\colon\mathbb R^n \to \mathbb R^m$. That is, for any $\vec{v}\in\mathbb R^n$ there exists an $\vec{u}\in\mathbb R^m$ such that $A\vec{v} = \vec{u}$ is satisfied and the property of linearity holds for any scalar from $\mathbb R$.

\section{Diagonalization}
\subsection{Eigenvalues and eigenvectors}
\begin{definition}
	Let $A$ be an $n\times n$-matrix, $\lambda \in \mathbb C$ and $v\in V$ be a nonzero vector. Then $\lambda$ is called the \emph{eigenvalue} of $A$ and $\vec{v}$ is called the \emph{eigenvector} of $A$ if $Av = \lambda v$ holds.
\end{definition}
Note that we can rearrange the above condition to the following
$$
Av = \lambda v \iff Av - \lambda v = 0 \iff \left(A - \lambda I\right)v = 0
$$
Since we require $v \neq 0$ it must be true that $A - \lambda I = 0$. It follows that the matrix $A - \lambda I$ is not invertible because $0$ is not invertible. It must also holds that $\det\left(A - \lambda I\right) = 0$.
\begin{definition}[Characteristic equation]
	Let $A$ be an $n\times n$ matrix. Then $\det\left(A - \lambda I\right) = 0$ is called the \emph{characteristic equation} of $A$.
\end{definition}
\begin{theorem}
	Let $A$ be an $n\times n$-matrix. Then $\det\left(A - \lambda I\right)$ is a polynomial of degree $n$.
	\begin{proof}
		By evaluating the first term of $\det\left(A - \lambda I \right)$ with the cofactor expansion, we obtain that
		$$
		\det\left(A - \lambda I \right) = \left[\prod_{i = 1}^n (a_{(i,i)} - \lambda)\right] + q(\lambda)
		$$
		It is clear here that the cofactor expansion yields a polynomial of degree $n$.
	\end{proof}
\end{theorem}
\begin{corollary}
	Let $A$ be an $n\times n$-matrix. Then $A$ has $n$ eigenvalues.
	\begin{proof}
		By the proceeding theorem, since $\det(A - \lambda I)$ is an polynomial of $n$-degree. It follows from the fundamental theorem of algebra that $\det(A - \lambda I)$ has $n$ solutions, i.e. $n$ eigenvalues.
	\end{proof}
\end{corollary}
We will not prove the fundamental theorem of algebra as that would require deeper knowledge in abstract algebra which is very redundant for the purpose of this paper. It also follows from the fundamental theorem of algebra that roots of the characteristic equation are elements of the set of complex numbers.
\begin{definition}[Similarity]
	Let $A$ and $B$ be two matrices. If there exists an invertible matrix $P$ such that $B = P^{-1} AP$ then  $A$ and $B$ are called \emph{similar}.
\end{definition}
\begin{definition}[Diagonalizable]
	Let $A$ be an $n\times n$-matrix and let $D$ be a diagonal $n\times n$-matrix. If $A$ and $D$ are similar then we say that $A$ is \emph{diagonalizable}.
\end{definition}
\begin{theorem}
	Let $A$ be an $n\times n$-matrix. Then $A$ is diagonalizable if and only if $A$ has $n$-linearly independent eigenvectors.
	\begin{proof}
		Let $P$ be the matrix given by
		$$
		P = \begin{pmatrix}
			| & | & & |\\
			v_1 & v_2 & \cdots & v_n\\
			| & | & & |
		\end{pmatrix}
		$$
		and let the diagonal matrix $D$ be
		$$
		D = \begin{pmatrix}
			\lambda_1 & 0 & \cdots & 0\\
			0 & \lambda_2 & \cdots & 0\\
			\vdots & \vdots & \ddots & 0\\
			0 & 0 & \cdots & \lambda_n
		\end{pmatrix}
		$$
		where $\lambda_i$ is the $i$-th eigenvalue of $A$ and $v_i$ is the corresponding eigenvector of $A$. It is easy to see that since $Av = \lambda v$. We can compactly express $A$ as $AP = PD$. It follows that $P$ is invertible if and only if the eigenvectors are linearly independent. Thus, if this is the case $A$ is diagonalizable $A = PDP^{-1}$.
	\end{proof}
\end{theorem}



\clearpage
	\chead{Probability Theory}
\part{Probability Theory}
\section{Lognormal Distribution}
\begin{definition}[Normal Distribution]
	A Continous random variable $X$ is normally distributed $X\sim\mathcal{N}(\mu, \sigma^2)$ with mean $\mu$ and variane $\sigma^2$ if the probability density function of $X$ is
	$$
	f(x) = \frac{1}{\sigma \sqrt{2\pi}}e^{-\frac{1}{2}\left(\frac{x - \mu}{\sigma}\right)^2}
	$$
	for all $x\in\mathbb R$.
\end{definition}
Now, for stock price analysis the normal distribution is not a good model for modelling the stock market, i.e. $P_n - P_{n-1} \sim \mathcal{N}(0, 1)$ is not a good model. Instead, we want the relative difference $\dfrac{P_n - P_{n-1}}{P_n} \sim \mathcal N(0, 1)$. The question that rises is what does the distribution of $P_n$ look like? We will hence, introduce the log-normal distribution.
\begin{definition}[Log-normal distribution]
	Let $X$ be a continuous random variable. Then $X$ is said to be \emph{lognormal} distributed if $Y = \ln X$ is normally distributed.
\end{definition}
In order to derive the distribution of $Y$ we can use the theorem of change of variables that states the following:
\begin{theorem}[Change of variable]
	Suppose that $X$ and $Y$ are random variables with pdf $f_X$ and $f_Y$ respectively, such that $P(X \leq x) = P(Y \leq h(x))$ for all $x$ then
	$$
	f_X(x) = \left(f_Y\circ h\right)(x)\frac{dh}{dx}
	$$
	\begin{proof}
		Consider the cumulative distribution $F_X(x) = F_Y(h(x))$. The proof then follows from the fundamental theorem of calculus that $f_X(x) = f_Y(h(x))h'(x)$.
	\end{proof}
\end{theorem}
From the proceeding definition if we let $X\sim\mathrm{Lognormal}(\mu, \sigma^2)$ and $Y = \ln X \sim\mathcal{N}(\mu,\sigma^2)$. Then by the proceeding theorem we have that
$$
f_X(x) = \frac{1}{x}f_Y(\ln(x)) = \frac{1}{\sigma\sqrt{2\pi}x}e^{-\frac{1}{2}\left(\frac{\ln(x)-\mu}{\sigma}\right)^2}
$$
Now that we have the pdf of $X$ it is natural to ask what $\expected$ and $\variance$ are respectively. Before we do that, it would be easier to derive $\expected$ and $\variance$ if we first introduce the moment generating function and investigate some of its properties first.
\begin{definition}[Moment generating function]
	Let $X$ be a random variable. Then the moment generating function (mgf) of $X$ is $M_X(t) = \expected[e^{tX}]$.
\end{definition}
\begin{theorem}
	Let $X\sim \mathcal{N}(\mu,\sigma^2)$ then
	$$
	M_X(t) = e^{t\mu + \frac{1}{2}\sigma^2t^2}
	$$
	\begin{proof}
		From the definition we have that
		$$
		M_X(t) = \expected[e^{tX}] = \frac{1}{\sqrt{2\pi}\sigma}\int_{-\infty}^\infty e^{tx}e^{-\frac{1}{2}\left(\frac{x - \mu}{\sigma}\right)^2}dx
		$$
		expanding the square gives us
		\begin{align*}
		tx - \frac{1}{2}\left(\frac{x - \mu}{\sigma}\right)^2 &= -\frac{x^2 - 2\mu x - 2\sigma^2tx + \mu^2}{2\sigma^2}\\
		&= - \frac{x^2 - 2(\mu + \sigma^2t)x + \mu^2 + (\mu + \sigma^2t)^2 - (\mu + \sigma^2t)^2}{2\sigma^2}\\
		&= - \frac{1}{2}\left(\frac{x - (\mu + \sigma^2t)}{\sigma}\right)^2 - \frac{\mu^2 - (\mu + \sigma^2t)^2}{2\sigma^2}\\
		&= - \frac{1}{2}\left(\frac{x - (\mu + \sigma^2t)}{\sigma}\right)^2 + \mu t + \frac{1}{2}\sigma^2 t^2
		\end{align*}
		Hence,
		$$
		M_X(t) = e^{\mu t + \frac{1}{2}\sigma^2 t^2}\cdot \frac{1}{\sqrt{2\pi}\sigma}\int_{-\infty}^{\infty}e^{- \frac{1}{2}\left(\frac{x - (\mu + \sigma^2t)}{\sigma}\right)^2} dx
		$$
		Since,
		$$
		\int_{-\infty}^{\infty}e^{- \frac{1}{2}\left(\frac{x - (\mu + \sigma^2t)}{\sigma}\right)^2} dx = \sqrt{2\pi}\sigma
		$$
		We have that
		$$
		M_X(t) = e^{\mu t + \frac{1}{2}\sigma^2 t^2}
		$$
	\end{proof}
\end{theorem}

\end{document}